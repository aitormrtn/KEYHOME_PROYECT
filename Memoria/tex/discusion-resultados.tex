\chapter{Discusión de resultados}
\label{ch:discusion-resultados}

Los resultados obtenidos a lo largo del desarrollo de este trabajo de fin de grado han sido en gran medida satisfactorios. El objetivo de automatizar por completo la reserva de un apartamento turístico ha sido llevado a cabo con éxito. Se presenta como resultado, por tanto, un producto funcional que ofrece un servicio adecuado y seguro para llevar a cabo la labor que es objeto de este proyecto.

No obstante, como todo avance tecnológico, la investigación y el desarrollo se han encontrado con aspectos positivos y también negativos, que vale la pena mencionar para poner en situación al lector y hacer que tenga una comprensión más profunda sobre el tema que se está tratando.

\section{Aspectos positivos}
\subsection{Instalación sencilla}
Uno de los aspectos positivos que vale la pena considerar es el hecho de que la instalación para el usuario final del dispositivo resultante de este proyecto tiene un carácter bastante sencillo. Únicamente deberá instalarse el aparato junto al portero automático conectado a un enchufe por medio de un cargador tipo USB micro B, disponible en multitud de comercios. El resto de la instalación consiste tan solo en conectar la hembra del conector mini jack en el telefonillo y poner cada uno de sus bornes en contacto con los puertos que permitan la apertura de la cerradura eléctrica desde el portero automático. Por ello, se obtiene un producto que puede instalarse en pocos minutos sin necesidad de conocimientos profundos en materia eléctrica o informática.
\subsection{Uso intuitivo}
El proyecto se ha desarrollado con éxito permitiendo tanto la gestión de la vivienda como el acceso a usuarios finales desde la aplicación de mensajería instantánea Telegram. El hecho de que sea la mensajería instantánea la que intermedie en todo momento hace que se presente un producto intuitivo y fácil de usar para multitud de usuarios que, en su vida diaria ya están acostumbrados a tratar con este tipo de aplicaciones.
\subsection{Producto seguro}
El resultado que se expone en este documento ha trabajado la seguridad en todos los aspectos posibles, desde su nivel más físico, impidiendo el acceso a la Raspberry Pi por medio de un diseño de impresión 3D protegido con candado, hasta sus niveles más sofisticados, avisando al administrador en caso de que se produzca una apertura o de que cualquier programa interrumpa su correcto funcionamiento. Este punto pone de manifiesto la importancia que se le da a la hora de proteger las residencias de aquellas personas que puedan interesarse por el producto.
\section{Aspectos negativos}
\subsection{Un mercado hermético}
Cabe destacar, sin embargo, que las posibilidades de aplicaciones que ampliarían la escalabilidad de este proyecto se ven comprometidas por la propia forma en que se estructura el mercado de las vivienda de uso turístico. En España, este negocio multimillonario es controlado y dirigido principalmente por 2 empresas (Booking y Airbnb) las cuales son bastante restrictivas a la hora de permitir trabajar con sus APIs.

Tan solo unas pocas empresas, que actúan como channel managers, tienen la posibilidad de trabajar con estos gigantes haciendo aplicaciones que actualicen los precios, gestionen las reservas, automaticen los registros de huéspedes para la policía, y otras muchas actividades que resultan de gran utilidad para los anfitriones. 

Este hecho provoca que la manera de trabajar para coger las reservas haya consistido, tal como se explica en el apartado de desarrollo del presente trabajo de fin de grado, en extraer la información necesaria de los correos electrónicos que notifican la información a los anfitriones. Este método, aunque se ha podido comprobar en este trabajo que funciona de manera correcta, no es la forma más eficiente de hacerlo, ya que podrían surgir problemas como el hecho de que las empresas de alquiler vacacional oculten información como las fechas de reserva o el número de reserva en los correos electrónicos, permitiendo tan solo acceder a los mismos desde el panel de administración de cada plataforma.

\subsection{Adaptación con cerraduras electromecánicas}
Una característica adicional que, por razones principalmente de coste, habría sido interesante llevar a cabo, sería la adaptación del sistema desarrollado con cerraduras inteligentes motorizadas. Activar la apertura con este tipo de cerraduras permitiría que no solo se emplearan cerraduras eléctricas en las viviendas donde se desee adaptar este sistema, si no que podría adaptarse a todo tipo de puertas y cerraduras, ofreciendo de esa manera, mayor seguridad ante robos o posibles allanamientos al usar infraestructuras más seguras.


