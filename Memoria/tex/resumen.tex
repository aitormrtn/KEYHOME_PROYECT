\begin{resumen}

\noindent Desde hace ya varios años, el auge de empresas tecnológicas asociadas al alquiler vacacional, como Booking\footnote{\url{https://www.booking.com/}} o Airbnb\footnote{\url{https://www.airbnb.es/}}, han propiciado un cambio significativo en la manera que funciona el turismo, dando la oportunidad, tanto a profesionales del sector como a principiantes, de tener visibilidad ante clientes potenciales.

Es un hecho que muchas personas han hecho de este modelo su forma de ganarse la vida, dedicándole a esta tarea su jornada laboral, a tiempo parcial o completo, para gestionar una o varias viviendas de uso turístico. Algunas de las tareas que se desempeñan en estos puestos de trabajo son las siguientes:
\begin{itemize}
\item Actualizar los precios en función de la demanda.
\item Responder las dudas a los clientes.
\item Ofrecer acceso a la vivienda y atender a los usuarios en su llegada.
\item Gestionar los trabajos de limpieza.
\item Prevenir el exceso de ventas para una misma habitación en el mismo día cancelando la disponibilidad de una misma vivienda en distintas plataformas.
\end{itemize}

\noindent En la actualidad existen herramientas que permiten actualizar los precios de forma automática en base a algoritmos que tienen en cuenta la demanda, y también pueden encontrarse algunas que permiten sincronizar los calendarios de las distintas plataformas entre sí, con el fin de facilitar el trabajo de los propietarios de viviendas de uso turístico.

El objeto de este trabajo de fin de grado consiste en gestionar la llegada del cliente a la vivienda de uso turístico, haciendo que las tareas desempeñadas entre el momento en que se realizan las reservas, y el que los clientes acceden a la vivienda, queden completamente automatizados, liberando de esta manera una importante carga de trabajo a los propietarios. El resultado de este trabajo es hacer que el anfitrión de la vivienda de uso turístico pueda tener la garantía de que su negocio se está gestionando de forma automática, y que solo se le avisará en casos particulares, cuando sea estrictamente necesario.

Para llevar a cabo esta tarea se creará un sistema que permita gestionar, de forma automática, por medio de una Raspberry Pi Zero W, la apertura de las puertas de acceso a la vivienda en los momentos que proceda. De igual manera, se integrarán en el sistema todas aquellas tareas que pueden tenerse en cuenta, como las posibles cancelaciones, la organización de los equipos de limpieza o las medidas de seguridad que garantizarán la tranquilidad de aquellos anfitriones que deseen hacer uso de este sistema.

Tanto los usuarios finales, como el administrador, podrán hacer uso de todas estas herramientas a partir de la aplicación de mensajería instantánea "Telegram". La elección de Telegram como aplicación de la plataforma no es casual, si no que responde a unos criterios bien definidos:
\begin{itemize}
\item Telegram es, y siempre ha sido, una aplicación de código abierto, lo cual abre la puerta a posibilidades infinitas de programación, así como la seguridad de saber que el sistema será estable en relación a las políticas de la plataforma. 
\item La inmensa mayoría de la población vive familiarizada con las aplicaciones de mensajería instantánea, por lo que la curva de aprendizaje sería mínima, y ello aceleraría considerablemente la adaptación de los clientes potenciales.
\item Telegram trabaja con un cifrado de 64 bits, de extremo a extremo, en todas sus comunicaciones. Esto facilita considerablemente el trabajo y resulta de gran utilidad a la hora de evitar interceptaciones de información por parte de terceros.
\end{itemize}
\end{resumen}

\begin{abstract}
\noindent For several years now, the rise of technology companies associated with vacation rental, such as Booking\footnote{\url{https://www.booking.com/}} or Airbnb\footnote{\url{https://www.airbnb.es/}}. They have brought about a significant change in the way tourism works, giving professionals, as well as beginners, the opportunity to have visibility before potential clients.

It is a fact that many people have made this model their way of earning a living, dedicating their part-time or full-time workday to this task to manage one or more houses for tourist use. Some of the tasks performed in these jobs are the following:
\begin{itemize}
\item Update prices based on demand.
\item Answer questions to customers.
\item Offer access to housing and attend to users upon arrival.
\item Manage cleaning jobs.
\item Prevent excess sales for the same room on the same day by canceling the availability of the same home on different platforms.
\end{itemize}

\noindent Currently there are tools that allow updating prices automatically based on algorithms that take into account demand, and there are also some that allow synchronizing the calendars of the different platforms with each other, in order to facilitate the work of homeowners for tourist use.

The purpose of this final degree project is to manage the client's arrival at the dwelling for tourist use, making the tasks performed between the time the reservations are made and the clients accessing the dwelling completely automated, thus relieving a significant workload for owners. The result of this work is to ensure that the host of the tourist-use dwelling can have the guarantee that his business is being managed automatically, and that he will only be notified in particular cases, when strictly necessary.

To carry out this task, a system will be created that automatically manages, by means of a Raspberry Pi Zero W, the opening of the access doors to the house at the appropriate times. Likewise, all those tasks that may be taken into account, such as possible cancellations, the organization of cleaning teams, or security measures that will guarantee the tranquility of those hosts who wish to use this system, will be integrated into the system.

Both end users and the administrator will be able to make use of all these tools from the "Telegram" instant messaging application. The choice of Telegram as application of the platform is not accidental, if not that it meets well-defined criteria:
\begin{itemize}
\item Telegram is, and always has been, an open source application, which opens the door to infinite programming possibilities, as well as the security of knowing that the system will be stable in relation to the platform's policies.
\item The vast majority of the population lives familiarized with the applications of instantaneous mail, reason why the learning curve would be minimal, and this would accelerate considerably the adaptation of the potential clients.
\item Telegram works with 64-bit encryption, end-to-end, in all its communications. This considerably facilitates the work and is very useful in avoiding interception of information by third parties.
\end{itemize}
\end{abstract}
